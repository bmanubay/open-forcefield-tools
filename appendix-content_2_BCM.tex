% appendix-content_2 has updated property values (provided by Ken Kroenlein).
% appendix-content has old property values

\documentclass[a4paper,12pt]{article}
\begin{document}
\appendix
\section{Property Derivations}
\subsection{Pure Solvent Properties}
\subsubsection{Density}
Starting with the equation used to calculate the density experimentally, 
\begin{equation} \rho = \frac{M}{\langle V \rangle} \end{equation}
Recall from the derivative form for the Gibbs free energy $dG = Vdp -SdT + \sum_i \mu_i dN_i$ that V can be found from the Gibbs free energy with
\begin{equation} V = \left( \frac{\partial G}{\partial p} \right)_{T,N} \end{equation}
The density can therefore be found from the Gibbs free energy.
\begin{equation} \rho = \frac{M}{ \left( \frac{\partial G}{\partial p} \right)_{T,N}} \end{equation}
The derivative can be estimated using a central difference numerical method utilizing Gibbs free energies reweighted to different pressures.
\begin{equation} \left( \frac{\partial G}{\partial p} \right)_{T,N} \approx \frac{G_{p + \Delta p} - G_{p-\Delta p}}{2\Delta p} \end{equation}
The density can be estimated.
\begin{equation} \rho \approx \frac{M}{\frac{G_{p + \Delta p} - G_{p-\Delta p}}{2\Delta p}} \end{equation}


\subsubsection{Dielectric Constant}
This equation was provided by a literature reference authored by CJ Fennell ("Simple Liquid Models with Corrected Dielectric Constants"). Below, $\epsilon(0)$ is the zero frequency dielectric constant, $V$ is the system volume and $M$ is the total system dipole moment. Note that the $\langle \rangle$ operator denotes an ensemble average for the system.

\begin{equation} \epsilon(0) = 1 + \frac{4 \pi}{3 k_B T \langle V \rangle}(\langle M^2 \rangle - \langle M \rangle^2) \end{equation}


%\begin{equation} \end{equation}
\subsubsection{Isothermal Compressibility}
The definition of isothermal compressibility is:
\begin{equation}\kappa_T = -\frac{1}{V} \left(\frac{\partial V}{\partial P}\right)_T \end{equation}

\begin{equation}\kappa_T = -\frac{\left(\frac{\partial^2 G}{\partial P^2}\right)_{T, N}}{\left(\frac{\partial G}{\partial P}\right)_{T, N}}\end{equation}\\*

$\kappa_T$ can also be estimated from the ensemble average and fluctuation of volume or particle number, thusly:
\begin{equation}\kappa_T = \beta \frac{\langle \Delta V^2 \rangle_{NTP}}{\langle V \rangle_{NTP}} = V \beta \frac{\langle \Delta N^2 \rangle_{VT}}{\langle N \rangle_{VT}}\end{equation}\\*

Dadarlat et al. "Insights into Protein Compressibility from MD Simulations" used as reference




\subsubsection{Molar Enthalpy}
Section on relation of enthalpy to Gibbs free energy (should we need it).
The enthalpy, $H$, can be found from the Gibbs free energy, $G$, by the Gibbs-Helmholtz relation: 

\begin{equation}H=-T^2 \left(\frac{\partial \big(\frac{G}{T}\big)}{\partial T}\right)_{P,N}\end{equation}

Transforming the derivative in the Gibbs-Helmholtz relation to be in terms of $\beta$ instead of $T$ yields:

\begin{equation}H=-T^2  \frac{\beta^2}{\beta^2}\left(\frac{\partial \big(\frac{G}{T}\big)}{\partial T} \frac{\partial T}{\partial \beta} \frac{\partial \beta}{\partial T}\right)_{P,N}\end{equation}


Recall that $\beta = \frac{1}{k_B T}$, therefore $\frac{\partial \beta}{\partial T} = - \frac{1}{k_B T^2}$. Substituting these values into the enthalpy equation gives:



\begin{equation}H = \frac{1}{k_B^3 T^2 \beta^2} \left(\frac{\partial \big(\frac{G}{T}\big)}{\partial \beta}\right)_{P,N} = \frac{1}{k_B} \left(\frac{\partial \big(\frac{G}{T}\big)}{\beta}\right)_{P,N}\end{equation}


Applying the quotient rule to the partial derivative yields


\begin{equation}H = \frac{T}{k_B} \left(\frac{\partial G}{\partial \beta}\right)_{P, N} - \frac{G}{k_B}\left(\frac{\partial T}{\partial \beta}\right)_{P, N} = \frac{1}{T k_B} \left(\frac{\partial G}{\partial \beta}\right) - \frac{G}{T^2 k_B} \left(\frac{\partial T}{\partial \beta}\right)\end{equation}


Recall that $\left(\frac{\partial T}{\partial \beta}\right)_{P, N} = -k_B T^2$, which allows the enthalpy to be simplified to:


\begin{equation} H = \beta \left(\frac{\partial G}{\partial \beta}\right)_{P, N} + G\end{equation}


The derivative can be estimated using a central difference numerical method utilizing Gibbs free energies reweighted to different temperatures.

\begin{equation} \left( \frac{\partial G}{\partial \beta} \right)_{P,N} \approx \frac{G_{\beta + \Delta \beta} - G_{\beta -\Delta \beta}}{2\Delta \beta} \end{equation}

%\begin{equation} \end{equation}
\subsubsection{Heat Capacity}
The definition of the isobaric heat capacity is:
\begin{equation}C_P =\left( \frac{\partial H}{\partial T}\right)_{P,N}\end{equation}

\begin{equation}C_P =  -k_B \beta^3 \left(\frac{\partial^2 G}{\partial \beta^2}\right)_{P,N}\end{equation}\\*


Finite difference approach? See section on Molar Enthalpy\\*

The form is equivalent for isochoric heat capacity, but with derivatives at constant volume rather than pressure.

\subsubsection{Speed of Sound}
The definition of the speed of sound is:
\begin{equation}c^2 = \sqrt{\left(\frac{\partial P}{\partial \rho}\right)_{S}} = -\frac{V^2}{M}\left(\frac{\partial P}{\partial V}\right)_{S}\end{equation}
\begin{equation}c^2 = \frac{V^2}{\beta M}\left[\frac{\left(\frac{\gamma_V}{k_B}\right)^2}{\frac{C_V}{k_B}} + \frac{\beta}{V \kappa_T}\right]\end{equation}\\*

Where:\\*
\begin{equation}\gamma_V = \left(\frac{\partial P}{\partial T}\right)_{V} = \frac{C_V}{T \left(\frac{\partial S}{\partial P}\right)_{V}}\end{equation}\\*

$\gamma_V$ is known as the isochoric pressure coefficient. $\kappa_T$ is the same isothermal compressibility from section A.1.3\\*

Lustig et al. "Direct molecular NVT simulation of isobaric heat capacity, speed of sound and Joule-Thomson Coefficient" used as reference\\*


\subsubsection{Enthalpy of Vaporization}
The definition of the enthalpy of vaporization is:
\begin{equation}\Delta H_{vap} = H_{gas} - H_{liq} = E_{gas} - E_{liq} + P(V_{gas} - V_{liq})\end{equation}\\*

If we assume that $V_{gas} >> V_{liq}$ and that the gas is ideal (and can therefore neglect kinetic energy terms):
\begin{equation}\Delta H_{vap} = E_{gas, potential} - E_{liq, potential} + R T\end{equation}\\*

Can make this a single simulation calculation if we assume the intramolecular energies between the phase changes are the same...


\subsection{Binary Mixture Properties} 
\subsubsection{Mass Density, Speed of Sound and Dielectric Constant}
The methods for these calculations are the same for a multicomponent system.

\subsubsection{Activity Coefficient}
Then the definition of the activity coefficient is:
\begin{equation}\mu_{i} = \left(\frac{\partial G}{\partial n_{i}}\right)_{T,P,n_{j \neq i}}\end{equation}\\*

Where $n_i$ refers to a molecule of component $i$ and $n_{j \neq i}$ refers to all molecules other than component $i$.\\*

\begin{equation}\mu_{i} = \mu^0_i + k_B T ln\left(x_i \gamma_i\right)\end{equation}\\*

Where $x_i$ is the mole fraction of component $i$ and $\gamma_i$ is the activity coefficient of component $i$. Rearrangement of the previous equation yields:
\begin{equation}\gamma_i = \frac{exp\left(\frac{\mu_i - \mu^0_i}{k_B T}\right)}{x_i}\end{equation}\\*

\subsubsection{Excess Molar Properties}
The general definition of an excess molar property can be stated as follows:
\begin{equation}y^{E} = y^{M} - \sum_{i} x_i y_i\end{equation}\\*

Where $y^E$ is the excess molar quantity, $y^M$ is the mixture quantity, $x_i$ is the mole fraction of component $i$ in the mixture and $y_i$ is the pure solvent quantity. In general, the simplest methods for calculating excess molar properties for binary mixtures will require three simulations. One simulation is run for each pure component and a third will be run for the specific mixture of interest.

\subsubsection{Excess Molar Heat Capcity and Volume}
Excess molar heat capacities and volume will be calculated using the  methods for the pure quantities in section A in combination with the general method for excess property calculation above.\\*

\subsubsection{Excess Molar Enthalpy}
Excess molar enthalpy can be calculated using the general relation of molar enthalpy as it relates to Gibbs Free Energy from section A and the generalized method of ecess molar property calculation above or by the following:
\begin{equation}H^E = \langle E^M \rangle + P V^E - \sum_{i} x_i \langle E_i \rangle\end{equation}\\*

Where $\langle \rangle$ denotes an ensemble average and $V^E$ is calculated using the general method of excess molar properties.\\* 








\end{document}
